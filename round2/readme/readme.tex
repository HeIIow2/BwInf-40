\documentclass[a4paper,10pt,ngerman]{scrartcl}
\usepackage{babel}
\usepackage[T1]{fontenc}
\usepackage[utf8x]{inputenc}
\usepackage[a4paper,margin=2.5cm,footskip=0.5cm]{geometry}

% Die nächsten drei Felder bitte anpassen:
\newcommand{\Aufgabe}{readme} % Aufgabennummer und Aufgabennamen angeben
\newcommand{\TeilnahmeId}{63175}                  % Teilnahme-ID angeben
\newcommand{\Name}{Lars Noack}             % Name des Bearbeiter / der Bearbeiterin dieser Aufgabe angeben


% Kopf- und Fußzeilen
\usepackage{scrlayer-scrpage, lastpage}
\setkomafont{pageheadfoot}{\large\textrm}
\lohead{\Aufgabe}
\rohead{Teilnahme-ID: \TeilnahmeId}
\cfoot*{\thepage{}/\pageref{LastPage}}

% Position des Titels
\usepackage{titling}
\setlength{\droptitle}{-1.0cm}

% Für mathematische Befehle und Symbole
\usepackage{amsmath}
\usepackage{amssymb}

% Für Bilder
\usepackage{graphicx}

% Für Algorithmen
\usepackage{algpseudocode}

% Für Quelltext
\usepackage{listings}
\usepackage{color}
\definecolor{mygreen}{rgb}{0,0.6,0}
\definecolor{mygray}{rgb}{0.5,0.5,0.5}
\definecolor{mymauve}{rgb}{0.58,0,0.82}
\lstset{
  keywordstyle=\color{blue},commentstyle=\color{mygreen},
  stringstyle=\color{mymauve},rulecolor=\color{black},
  basicstyle=\footnotesize\ttfamily,numberstyle=\tiny\color{mygray},
  captionpos=b, % sets the caption-position to bottom
  keepspaces=true, % keeps spaces in text
  numbers=left, numbersep=5pt, showspaces=false,showstringspaces=true,
  showtabs=false, stepnumber=2, tabsize=2, title=\lstname
}
\lstdefinelanguage{JavaScript}{ % JavaScript ist als einzige Sprache noch nicht vordefiniert
  keywords={break, case, catch, continue, debugger, default, delete, do, else, finally, for, function, if, in, instanceof, new, return, switch, this, throw, try, typeof, var, void, while, with},
  morecomment=[l]{//},
  morecomment=[s]{/*}{*/},
  morestring=[b]',
  morestring=[b]",
  sensitive=true
}

% Diese beiden Pakete müssen zuletzt geladen werden
\usepackage{hyperref} % Anklickbare Links im Dokument
\usepackage{cleveref}

% Daten für die Titelseite
\title{\textbf{\Huge\Aufgabe}}
\author{\LARGE Teilnahme-ID: \LARGE \TeilnahmeId \\\\
	    \LARGE Bearbeiter/-in dieser Aufgabe: \\ 
	    \LARGE \Name\\\\}
\date{\LARGE\today}

\begin{document}

\maketitle


\vspace{0.5cm}

\section{Vorab}

Okay ich habe 2 Aufgaben bearbeitet. Rechenrätsel und Hex-Max. Eigentlich wollte ich aber die Extraufgabe Zara Zackigs Zurückkehr machen. Dass habe ich alles implementiert, aber die Laufzeit war sehr schlecht. Deshalb habe ich nach zwei Wochen eine andere Aufgabe gemacht. Ein konzept dass ich aus Zeitgründen nicht implementieren konnte wäre eine meet in the middle attack. Dies würde vermutlich funktionieren.

\section{Die Ordnerstruktur}

In dem geschickten Ordner befinden sich zwei Unterordner. Hex-Max und Rechenrätsel. In jedem dieser ordner ist eine pdf. Dies ist die Dokumentation. Auch ist in jedem Ordner zusätzlich ein Diagramm der praktischen Laufzeitanalyse zu finden, welches jedoch auch in der Dokumentation ist. In dem Ordner solutions sind jeweils diverse Beispiele und deren Lößungen. Das ausführbare programm ist in dem programm Ordner zu finden.

\subsection{Rechenrätsel}

\subsection{Hex Max}

Der skript 'laufzeit.py' berechnet und rendert lediglich das Diagramm. Zum ausführen wird generell lediglich numpy, matplotlib und pillow benötigt die alle wenn nötig mit pip installiert werden können.



\end{document}
